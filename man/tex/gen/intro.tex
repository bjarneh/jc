
\section{ Intro }

The setup required to use  \href{http://ant.apache.org}{Ant} or
 \href{http://maven.apache.org}{Maven} can be a bit too much for
small/medium sized Java projects. Naturally the JVM startup time,
and Ant/Maven's shared love of XML are also pretty good reasons
for an alternative approach.




\subsection{ Convention }

For convention to replace configuration something has to be said about
the conventions; hopefully this will be similar to how most Java projects
are organized anyway. Assuming we are starting project \texttt{x}, this setup
should do.


{\small
\begin{verbatim}

x/
 `-- src/      # source code is placed here

\end{verbatim}
}


After you have written some code and placed it the \texttt{src} folder, you
could do something like this assuming you have \texttt{jc} installed, or the
\texttt{jc} file itself could be placed into the \texttt{x} directory and you could
type \texttt{./jc} instead.


{\small
\begin{verbatim}

user@host:~/project/x$ jc
compiling: 1/1 packages
zip it up: dst/a.jar
time used: 0.736 seconds

\end{verbatim}
}


After your project finished compiling you will see that two new directories
have been created: \texttt{obj} and \texttt{dst}.


{\small
\begin{verbatim}

x/
 |-- src/      # still just source here..
 |-- obj/      # class files end up here
 `-- dst/      # destination of jar archive

\end{verbatim}
}


\texttt{obj} contains the compiled java-classes, and the \texttt{dst} directory holds a
jar/zip archive where the java-classes from the \texttt{obj} directory and a
\texttt{MANIFEST.MF} file (with some build info) are contained in side \texttt{a.jar}.



{\small
\begin{verbatim}

x/
 |-- src/      # source code goes here
 |-- lib/      # dependencies goes here
 |-- obj/      # class files end up here
 |-- dst/      # jar archive ends up in this directory
 |-- htm/      # html for Javadoc ends up here
 `-- res/      # resources to be included into jar goes here

\end{verbatim}
}

