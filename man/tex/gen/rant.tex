
\section{ Rant }

This chapter can be skipped, it's just personal opinions.



\subsection{ Ant }

 \href{http://ant.apache.org}{Ant} is one of the older build tools for Java, and
the default way to build projects for most IDE's, it can be configured to
do just about anything. My only problem with Ant is that is its
configuration has become nothing more than a massive template I keep
dragging from project to project.  I typically place my dependencies in a
directory called \texttt{lib}, and add them to my \texttt{CLASS\_PATH} etc.  I've never
remembered how to specify these targets inside the \texttt{build.xml}, so I need
my template or google nearby.  But since my template has stayed basically
the same for years, what I really wanted was some tool that did not require
any configuration for a plain setup, and if I had to do any
configuration; it should not be done with \textit{XML}.




\subsection{ Maven }

 \href{http://maven.apache.org}{Maven} is newer than Ant and ships with more
features out of the box, some which are \textbf{very} attractive at first sight.

\begin{itemize}

\item{ Calculates and downloads recursive dependencies}

\item{ Little configuration needed for typical setup (relatively)}

\end{itemize}


I started using Maven a while back and I have always been sort of content
with it; perhaps not the fastest tool around, but it did what was advertised.
It's only after I started using it for work purposes that I
started to realize the mess you can end up in with this tool; I'll try
to give a summary.



\subsection{ Dependencies }

It sounds like a good idea; all you have to do is to specify a dependency
and its version and voilà, the package and all its recursive dependencies are
fetched for you, but fist a couple of questions:

\begin{enumerate}

\item{ Have you ever built a Maven project without knowing what you depend on?}

\item{ Do you regularly check your dependencies and their recursive dependencies to make sure they are up to date with the latest version of the library?}

\end{enumerate}


If you answer \textbf{yes} on both of these questions, then Maven is clearly
the tool for you, but in my experience this practically never happens.



\subsection{ Repositories }

Naturally Maven fetches its dependencies from repositories, but what has
really been making a mess out of projects I've come across in the last year
is that these repositories are not static. Where I work there are several
projects that will not build \textit{unless} you copy your coworkers
\texttt{\$HOME/.m2} directory over to your machine first. The repositories where
these dependencies stem from are dead, or the packages we are looking for
(in that version) is nowhere to be found. Was it removed because
it was \textit{buggy/insecure/awful}? Who knows; it's gone and nobody knows what
that library was up to, it's was a recursive dependency far down the
rabbit-hole of some massive project.



\subsubsection{ Private Repositories }

In order to avoid copying coworkers \texttt{\$HOME/.m2} we naturally created
private repositories, basically we have our own \textbf{graveyard} repository of
all the dead libraries we depend on, not exactly an ideal solution.



\subsubsection{ Dynamic Repositories }

The fact that repositories modify their content and fail to contain
packages forever more cannot be blamed on Maven. The only fault Maven
has in all of this is their treatment of dependencies, Maven treats
dependencies as nothing but a build-requirement that \textit{it} can  take care
of.  This quickly creeps into your mindset and pretty soon you sit there
scratching your head like an  IDE-user who's baffled about loosing
oversight on a project. That is what IDE's are good at, \textbf{giving you a
false sense of oversight}, just like an automated dependency fetcher is
good at distancing you from the mess you import into your project.



\subsubsection{ Technical? }

Maven's biggest weakness in my opinion is not technical, it's political;
i.e. I think it leads to thoughtless dependencies and zero quality
control of the source imported into a project. The best way to solve this
problem is \textbf{not} to solve it; since there is no way of doing this
automatically and keep any form of discipline regarding imports.



\subsection{ Captain Slow }

Any Java based project must be forgiven for a bit of overhead with the
JVM startup and so on, but with Maven, it's not just a little overhead
that can be forgiven; I'll illustrate with an example.


\subsubsection{ Example }


{\small
\begin{verbatim}

user@host:cop (hg) $ time mvn compile
[INFO] Scanning for projects...
[INFO] ------------------------------------------------------
[INFO] Building cop
[INFO]    task-segment: [compile]
[INFO] ------------------------------------------------------
[INFO] [resources:resources {execution: default-resources}]
[INFO] Using 'UTF-8' encoding to copy filtered resources.
[INFO] [compiler:compile {execution: default-compile}]
[INFO] Nothing to compile - all classes are up to date
[INFO] ------------------------------------------------------
[INFO] BUILD SUCCESSFUL
[INFO] ------------------------------------------------------
[INFO] Total time: 2 seconds
[INFO] Finished at: Thu Mar 21 00:42:22 CET 2013
[INFO] Final Memory: 13M/85M
[INFO] ------------------------------------------------------

real    0m3.957s
user    0m6.060s
sys     0m0.312s
\end{verbatim}
}


The project isn't very big, a line count of about 6400, and sloc of
4138 according to David A. Wheeler's SLOCCount. My computer is a Thinkpad
X301 with a 1.4 GHz dual core processor. What really bugs me is this:


{\small
\begin{verbatim}
[INFO] Nothing to compile - all classes are up to date
\end{verbatim}
}


combined with this:


{\small
\begin{verbatim}
$ time javac -d target/classes $(find src/main -name '*.java')

real    0m1.590s
user    0m2.012s
sys     0m0.128s
\end{verbatim}
}


It's actually \textbf{faster to build the entire project twice}, than for Maven
to calculate that \textbf{nothing} has to be done, \textit{all classes are up to date}.



\subsubsection{ What? }

If you did not look closely, you may have missed a pretty strange detail
in the previous section, look at this once again:


{\small
\begin{verbatim}
$ javac -d target/classes $(find src/main -name '*.java')
\end{verbatim}
}


Did \texttt{find} magically spit out the source code in the correct build order
or does \texttt{javac} actually sort them? I.e. if \texttt{javac} calculates build order
there really isn't much left to do for a build tool, part from checking
timestamps to see which files are up to date.



\subsubsection{ \texttt{javac} }

As we just saw; \texttt{javac} is \textit{almost} a build tool.

